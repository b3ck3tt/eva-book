%!TEX root = main.tex

\emph{Evluční strategie}\cite{rechenberg1973,schwefel1977numerische} jsou z historického hlediska jistou alternativou Hollandova genetického algoritmu. Jsou o něco starší a je na nich zajímavé, že jsou o něco komplikovanější. V současnosti se používají především pro řešení problémů spojité optimalizace, ale myšlenky, které se v této oblasti vyskytují, se dají použít i jinde.

Evoluční strategie se dělí do dvou skupin podle způsobu, jakým pracují s populacemi rodičů a potomků. V obou případech jsou důležité parametry $\mu$ a $\lambda$, které označují počet rodičů a počet potomků, kteří z nich vznikají. Pro druhy evolučních strategií potom existuje ustálené značení $(\mu, \lambda)$-ES a $(\mu + \lambda)$-ES. V prvním případě (``čárková selekce'') máme populaci $\mu$ rodičů, ze kterých vytvoříme $\lambda$ potomků ($\lambda > \mu$), z těch potom vybereme nejlepších $\mu$ jako rodiče do další generace.  Ve druhém případě (``plus selekce'') z $\mu$ rodičů vytvoříme opět $\lambda$ potomků, ale před selekcí napřed sloučíme rodiče a potomky do jedné populace velikosti $\mu + \lambda$. Z té se potom opět vybere $\mu$ nejlepších jedinců jako rodiče do další generace. 

Jednou ze základních vlastností evolučních strategií, které je odlišují od jiných typů evolučních algoritmů je to, že obsahují nějakou formu samo-adaptace parametrů. V případě spojité optimalizace se tedy kromě samotných hodnot vektoru vyvíjí například i parametry pro mutaci. Technicky se tedy potom jedinec skládá ze dvou částí -- samotného zakódovaného vektoru čísel $\vec{x}$ a vektoru tzv. \emph{endogenních parametrů} $\vec{s}$, které právě obsahují všechny parametry, které ovlivňují chování operátorů\footnote{Vedle pojmu ``endogenní parametry'' se ještě občas objevuje pojem ``exogenní parametry'', který označuje parametry algoritmu, které se nemění, jako např. velikost populace.}. Typickým příkladem může být jedinec, který kromě řešení obsahuje i rozptyl gaussovské mutace, který se použije, pokud je tento jedinec mutován. Je důležité si uvědomit, že endogenní parametry nijak přímo neovlivňují fitness jedince, jejich hodnoty se vyvíjí jen díky tomu, že jedinci s lepší hodnotou endogenních parametrů mají po aplikaci genetických operátorů častěji lepší fitness. Pro samotnou evoluci endogenních parametrů se typicky používají stejné operátory jako pro samotného jedince s fixně nastavenými parametry.