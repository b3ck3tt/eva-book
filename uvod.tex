
Prohledávání prostoru řešení parametricky zadaných úloh je jedním z hlavních problémů mnoha oblastí informatiky i matematiky. Aplikace prohledávání jsou nesčetné a zahrnují optimalizaci reálných parametrů, kombinatorickou optimalizaci, automatický vývoj programů, učení robotů nebo návrh architektury a učení neuronových sítí. Přírodou inspirované algoritmy přinesly do tohoto oboru několik nových přístupů, které jsou schopny efektivně pracovat i ve vysoce dimenzionálních prostorech, optimalizovat nelineární funkce s lokálními extrémy a pracovat v tak strukturovaných prostorech jako jsou syntaktické stromy, grafy nebo neuronové sítě. 

Výhodou většiny přístupů zmíněných v tomto článku je jejich obecnost. Evoluční i rojové techniky můžeme chápat jako metaheuristiky, které nevyžadují mnoho specifických informací o řešené úloze, kromě ohodnocení jednotlivých kandidátů řešení prostřednictvím účelové funkce nazývané též ohodnocovací funkce nebo fitness. Výhodou tohoto přístupu je snadná aplikovatelnost generického algoritmu na široké spektrum úloh. Na druhou stranu, existují teoretické i praktické důvody pro přizpůsobení obecných prohledávacích heuristik dané úloze. Takto přizpůsobené metody obvykle dosahují lepších výsledků a zároveň si uchovávají obecné výhody těchto technik, jako je odolnost proti uváznutí v lokálních extrémech. 

Hlavním rysem zde uváděných přírodou inspirovaných prohledávacích technik je populační přístup. Na rozdíl od většiny algoritmů lokálního prohledávaní, které zkoumají nejbližší okolí jednoho bodu v prohledávacím prostoru, evoluční techniky pracují s populací desítek až tisíců řešení, které paralelně prohledávají prostor řešení a navzájem se ovlivňují. 

Lokální prohledávací metody šité na míru určitému problému typicky mají výhodu rychlejšího lokálního prohledávání díky využití specifických informací o problému, jako je například informace o gradientu účelové funkce. V dalším textu se zmíníme také o možnosti hybridizovat některé evoluční algoritmy pomocí specifického lokálního prohledávání. Tato technika se v~praxi osvědčuje zrychlením konvergence algoritmu, i když někdy přináší větší nebezpečí uváznutí v lokálních extrémech účelové funkce. Je zajímavé, že z~hlediska původní biologické inspirace představuje tato hybridizace překročení darwinistického rámce a přináší lamarckistické či epigenetické prinicpipy. 

\redo{update konec uvodu}
V následujícím textu nejprve stručně zmíníme inspiraci a obecné rysy evolučních algoritmů a pak se budeme věnovat několika konkrétním oblastem vycházejících z těchto obecných principů. Nejprve popíšeme tradiční genetické algoritmy pracující původně nad binárně zakódovanými jedinci. Dále budeme hovořit o evolučním programování, oblasti která stírá rozdíl mezi genotypem (zakódováním jedince pro účely evolučního prohledávání) a fenotypem (vlastním modelem, který vznikne dekódováním genotypu). Evoluční strategie byly prvním přístupem specializujícím se na optimalizaci reálných parametrů a také přinesly první koncept meta-evoluce, optimalizace parametrů evoluce pomocí vlastního evolučního algoritmu. Genetické programování je příkladem úspěšné evoluce složitých struktur syntaktických stromů počítačových programů. Neuroevoluce ukazuje možnosti využití evolučních technik pro určení struktury i parametrů modelů neuronových sítí. Oblast rojových algoritmů se inspiruje chováním společenského hmyzu a hejn pro efektivní prohledávání různých prostorů, např. reálných euklidovských prostorů nebo hledání cest v grafech. 
