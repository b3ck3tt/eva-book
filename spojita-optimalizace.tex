%!TEX root = main.tex

Mnoho optimalizačních problémů z běžného života se dá definovat jako optimalizace funkce $$f: \Real^n \to \Real\,.$$ Je proto přirozené, že mnoho výzkumníků se zabývá právě evolučními algoritmy, kterou jsou schopné optimalizovat takové funkce. Problém optimalizace takových funkcí se v literatuře objevuje pod pojmem \emph{spojitá optimalizace}, nebo anglicky \emph{continuous optimization}. Je důležité si uvědomit, že ten pojem vyjadřuje pouze to, že prostor, ve kterém se hledají řešení je spojitý ($\Real^n$), samotná optimalizovaná funkce $f$ být spojitá nemusí.

\section{Vlastnosti funkcí}

Je zřejmé, že některé typy funkcí budou pro evoluční algoritmy lehčí, než jiné. Velký vliv na efektivitu evolučního algoritmu mají především vlastnosti jako multi-modalita, separabilita a podmíněnost. 

Funkce je \emph{multi-modální}, pokud má velké množství lokálních optim. Je zřejmé, že v takovém případě může mít algoritmus problém s uváznutím v lokálním optimu a je potřeba tomu přizpůsobit operátory. Existuje i oblast multi-modální optimalizace, kde je cílem najít co nejvíce různorodých lokálních optim. 

Separabilní funkce jsou naopak pro optimalizaci jednodušší. Jsou to takové funkce, které se dají zapsat pomocí funkcí jedné proměnné. Formálně, funkce $f(x_1, \dots, x_n): \Real^n \to \Real$ je \emph{aditivně separabilní}, pokud se dá zapsat jako součet funkcí $f_1(x_1), \dots, f_n(x_n)$, tj. $f(x_1, \dots, x_n)= \sum_{i=1}^n f(x_i)\,.$ Obdobně můžeme zadefinovat i funkci multiplikativně separabilní. Z hlediska optimalizace je velkou výhodou separabilních funkcí, že se dají optimalizovat po jednotlivých složkách vektoru, tj. optimum můžeme najít tak, že vždy zafixujeme hodnoty $n-1$ parametrů a optimalizujeme jen podle jednoho.

\add{Dopsat definici podmíněnosti}

\begin{marginfigure}
\centering
\begin{tikzpicture}
\begin{axis}[
  axis x line=center,
  axis y line=center,
  xmin=-5,   xmax=5,
  ymin=-5,   ymax=5,
  xlabel=$x_1$,
  ylabel=$x_2$,
  width=\marginparwidth
]
  \pgfplotsinvokeforeach{0.2, 0.4, 0.6, 0.8, 1.0}
  {\draw[red] \pgfextra{
    \pgfpathellipse{\pgfplotspointaxisxy{0}{0}}
    {\pgfplotspointaxisdirectionxy{#1*4.5}{#1*1.8}}
    {\pgfplotspointaxisdirectionxy{0}{#1}}
  };}
\end{axis}
\end{tikzpicture}
\caption{Příklad špatně podmíněné funkce}
\end{marginfigure}