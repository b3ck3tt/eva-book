%!TEX root = main.tex

Mnoho optimalizačních problémů z běžného života se dá definovat jako optimalizace funkce $$f: D \to \Real\,,$$ kde $D \subseteq \Real^d$. Je proto přirozené, že mnoho výzkumníků se zabývá právě evolučními algoritmy, kterou jsou schopné optimalizovat takové funkce. Problém optimalizace takových funkcí se v literatuře objevuje pod pojmem \emph{spojitá optimalizace}\footnote{anglicky \emph{continuous optimization}}. Je důležité si uvědomit, že název
 vyjadřuje pouze to, že prostor, ve kterém se hledají řešení je spojitý ($\Real^n$), samotná optimalizovaná funkce $f$ být spojitá nemusí.

Velmi často je definiční obor funkce $D$ $d$-rozměrný interval $$D = [l_1, u_1] \times \dots \times [l_d, u_d]\,,$$ kde $l_i$ a $u_i$ jsou dolní a horní meze pro $i$-tou proměnnou. Objevují se ale i obecnější problémy, kde je $D$ dána pomocí množiny podmínek tvaru $g(\vec{x}) \leq 0$ a $h(\vec{x}) = 0$, pro $g, h: \Real^d \to \Real$.

V této kapitole napřed budeme uvažovat optimalizační problém tvaru 
\begin{equation*}
  \begin{aligned}
  & \underset{x}{\text{min}} & & f(\vec{x}) \\
  & \text{za podmínek} & & \vec{x} \in [l_1, u_1] \times \dots \times [l_d, u_d]\,. \\
  \end{aligned}
\end{equation*}

\section{Vlastnosti funkcí}

Je zřejmé, že některé typy funkcí budou pro evoluční algoritmy lehčí, než jiné. Velký vliv na efektivitu evolučního algoritmu mají především vlastnosti jako multi-modalita, separabilita a podmíněnost. 

Funkce je \emph{multi-modální}, pokud má velké množství lokálních optim. Je zřejmé, že v takovém případě může mít algoritmus problém s uváznutím v lokálním optimu a je potřeba tomu přizpůsobit operátory. Existuje i oblast multi-modální optimalizace, kde je cílem najít co nejvíce různorodých lokálních optim. 

Separabilní funkce jsou naopak pro optimalizaci jednodušší. Jsou to takové funkce, které se dají zapsat pomocí funkcí jedné proměnné. Formálně, funkce $f(x_1, \dots, x_n): \Real^n \to \Real$ je \emph{aditivně separabilní}, pokud se dá zapsat jako součet funkcí $f_1(x_1), \dots, f_n(x_n)$, tj. $f(x_1, \dots, x_n)= \sum_{i=1}^n f(x_i)\,.$ Obdobně můžeme zadefinovat i funkci multiplikativně separabilní. Z hlediska optimalizace je velkou výhodou separabilních funkcí, že se dají optimalizovat po jednotlivých složkách vektoru, tj. optimum můžeme najít tak, že vždy zafixujeme hodnoty $n-1$ parametrů a optimalizujeme jen podle jednoho.

Další vlastností, které výrazně ovlivňuje evoluci je podmíněnost funkce. Ta vyjadřuje, jak moc se liší vliv jednotlivých proměnných na hodnotu funkce. Pro kvadratické funkce (u kterých vrstevnice vypadají jako elipsoidy), je jejich podmíněnost druhá odmocnina poměru mezi délkou nejdelší a nejkratší osy elipsoidu. Pokud je podmíněnost funkce velká, říká se, že je funkce špatně podmíněná. Z hlediska evolučního algoritmu je důležité, že by s různým vlivem proměnných na hodnotu funkce měly počítat operátory.

\begin{marginfigure}
\centering
%!TEX root = ../../main.tex
\subfigure[Dobře podmíněná separabilní funkce]{
  \begin{tikzpicture}
  \begin{axis}[
    axis x line=center,
    axis y line=center,
    xmin=-5,   xmax=5,
    ymin=-5,   ymax=5,
    xlabel=$x_1$,
    ylabel=$x_2$,
    width=\marginparwidth
  ]
    \pgfplotsinvokeforeach{0.2, 0.4, 0.6, 0.8, 1.0}
    {\draw[red] \pgfextra{
      \pgfpathellipse{\pgfplotspointaxisxy{0}{0}}
      {\pgfplotspointaxisdirectionxy{#1*4.5}{0}}
      {\pgfplotspointaxisdirectionxy{0}{#1*4.5}}
    };}
  \end{axis}
  \end{tikzpicture}
}
\subfigure[Špatně podmíněná funkce]{
  \begin{tikzpicture}
  \begin{axis}[
    axis x line=center,
    axis y line=center,
    xmin=-5,   xmax=5,
    ymin=-5,   ymax=5,
    xlabel=$x_1$,
    ylabel=$x_2$,
    width=\marginparwidth
  ]
    \pgfplotsinvokeforeach{0.2, 0.4, 0.6, 0.8, 1.0}
    {\draw[red] \pgfextra{
      \pgfpathellipse{\pgfplotspointaxisxy{0}{0}}
      {\pgfplotspointaxisdirectionxy{#1*4.5}{0}}
      {\pgfplotspointaxisdirectionxy{0}{#1}}
    };}
  \end{axis}
  \end{tikzpicture}
}
\subfigure[Špatně podmíněná a neseparabilní funkce]{
  \begin{tikzpicture}
  \begin{axis}[
    axis x line=center,
    axis y line=center,
    xmin=-5,   xmax=5,
    ymin=-5,   ymax=5,
    xlabel=$x_1$,
    ylabel=$x_2$,
    width=\marginparwidth
  ]
    \pgfplotsinvokeforeach{0.2, 0.4, 0.6, 0.8, 1.0}
    {\draw[red] \pgfextra{
      \pgfpathellipse{\pgfplotspointaxisxy{0}{0}}
      {\pgfplotspointaxisdirectionxy{#1*4.5}{#1*1.8}}
      {\pgfplotspointaxisdirectionxy{0}{#1}}
    };}
  \end{axis}
  \end{tikzpicture}
}
\caption{Příklady různých vlastností funkcí}
\label{fig:function_examples}
\end{marginfigure}

Příklady posledních dvou vlastností vidíme na obrázku \ref{fig:function_examples}, který ukazuje vrstevnice funkcí dvou proměnných. Horní funkce je jednoduchý dvoudimenzionální paraboloid, který je separabilní a dobře podmíněný. Na prostředním obrázku je paraboloid, který má jednu osu delší než druhou a znázorňuje tedy špatnou podmíněnost\footnote{podmíněnost zobrazené funkce jen cca 4.8, nedá se tedy považovat za špatně podmíněnou, objevují se i funkce s podmíněností $10^6$}. Poslední funkce potom kombinuje špatnou podmíněnost s neseparabilitou.

Výše uvedené vlastnosti jsou ty, které nejvíce ovlivňují efektivitu evolučních algoritmů při spojité optimalizaci, nejsou to ale všechny. Některé algoritmy například mohou využívat různé symetrie dané funkce, naopak funkce, které mají své globální optimum jen ve velmi malé oblasti prohledávaného prostoru a jinak jsou konstantní jsou pro evoluci velmi těžké obecně.

\section{Kódování pro spojitou optimalizaci}

Jedno z prvních rozhodnutí, které je potřeba udělat při návrhu evolučního algoritmu je výběr kódování jedince. Vzhledem k tomu, že ve spojité optimalizaci pracujeme s vektory reálných čísel, je otázka výběru kódování relativně snadná a jedinci jsou v naprosté většině případů kódování jako vektor typu \code{float} nebo \code{double}.


\ask{Existuje pro tohle $\downarrow$ nějaká reference?}

Dalo by se uvažovat i o kódování jedince přímo po vzoru Hollandova genetického algoritmu, tj. binárním vektorem, a používat jednoduchá $n$-bodová křížení a bit-flip mutace, ale taková reprezentace trpí tím, že změna různých bitů v číslech vede k výrazně různým změnám hodnoty (např. změna bitu na konci mantisy vs. změna bitu na začátku exponentu).

\section{Operátory pro spojitou optimalizaci}

Pro spojitou optimalizaci můžeme samozřejmě použít stejné operátory jako pro každé jiné vektorové kódování jedince, tedy například $n$-bodové, případně uniformní křížení. Nicméně častěji se používají operátory specializované, které přímo využívají toho, že jedinci jsou vektory čísel.

Tzv. aritmetické křížení vektorů počítá vážený průměr dvou rodičů tak, aby vytvořilo potomka, potomci se tedy spočítají jako
\begin{align*}
o_1 & = w*p_1 + (1-w)* p_2\,, \\
o_2 & = (1-w)*p_1 + w* p_2\,,
\end{align*}
kde $p_1$ a $p_2$ jsou rodiče a $w \in (0,1)$ je (případně náhodně) zvolená váha. Ačkoliv se takové křížení relativně často používá hlavně v jednodušších aplikacích, jeho velkou nevýhodou je, že výslední potomci se nikdy nemohou dostat z konvexního obalu počáteční populace.

Mutace pro spojitou optimalizaci se občas rozdělují na dva typy --- ovlivněné a neovlivněné. Neovlivněná mutace generuje novou hodnotu pro složku vektoru nezávisle na aktuální hodnotě, ovlivněná mutace naopak aktuální hodnotu používá. Typickým příkladem neovlivněné mutace je vygenerování nového čísla z daného rozsahu. Ovlivněná mutace typicky přičítá k dané složce vektoru číslo z normálního rozdělení $\normal{\mu}{\sigma^2}$.

Jednou z nevýhod výše uvedeného aritmetického křížení je, že potomci jsou velmi často relativně daleko od svých rodičů a nejsou jim tedy moc podobní. Právě z toho důvodu Deb a Agrawal navrhli simulované binární křížení (SBX)\cite{agrawal1995simulated}. Hlavní inspirací pro vytvoření SBX bylo jednobodové křížená binárních řetězců, které reprezentují čísla ve dvojkové soustavě. Při náhodné volbě bodu pro křížení velká část nových potomků je relativně blízko k jednomu z rodičů. Jednobodové křížení má navíc další pěknou vlastnost -- oba potomci jsou stejně daleko od průměru rodičů. 

Cílem SBX je právě simulovat velikosti změn, které se dějí při jednobodovém křížení binárních řetězců. Noví potomci se v SBX křížení spočítají jako 
$$o_{1,2} = \frac{1}{2}(p_1 + p_2) \pm \frac{1}{2}\beta(p_2 - p_1)\,,$$ kde $p_1$ a $p_2$ jsou rodiče a $\beta$ je náhodné číslo s pravděpodobnostním rozdělením
$$P(\beta) = \twopartdef{\frac{1}{2} (n+1) \beta^n}{\beta \leq 1}
                        {\frac{1}{2} (n+1) \frac{1}{\beta^{n+2}}}{\beta>1\,,}$$
kde $n$ je parametr rozdělení, který určuje, jak často budou hodnoty $\beta$ blízko $1$. Pro vyšší hodnoty $n$ je tato pravděpodobnost vyšší (viz Obrázek~\ref{fig:sbx_beta}). 

\begin{marginfigure}[-12\baselineskip]
\centering
%!TEX root = ../../main.tex
\begin{tikzpicture}
  \begin{axis}[
    axis lines=middle,
    xmin=0,   xmax=5,
    ymin=0,   ymax=2,
    xlabel=$\beta$,
    ylabel=$P(\beta)$,
    xtick={0,1,2,3,4,5},
    ytick={0,1,2},
    width=\marginparwidth,
    vasymptote=1,
    legend pos=north east
  ]
  \addplot [domain=0:1,samples=100]{3/2*x^2};
  \addplot [domain=1:5,samples=100]{3/2*(1/x^4)};
  \addplot [domain=0:1,samples=100,color=red]{11/2*x^10};
  \addplot [domain=1:5,samples=100,color=red]{11/2*(1/x^12)};
  
  \node[anchor=west,color=red] (n10) at (axis cs:1.1,1.8){$n=10$};
  \node[anchor=west,color=black] (n2) at (axis cs:1.7,0.3){$n=2$};
  \end{axis}
\end{tikzpicture}
\caption{Distribuce $\beta$ v SBX křížení pro $n=2$ a $n=10\,.$}
\label{fig:sbx_beta}
\end{marginfigure}

Podobnou motivaci jako SBX má i tzv. \emph{polynimální mutace} (PM)\cite{Deb96acombined}. V tomto případě se simuluje velikost změn v bit-flip mutaci binárně kódovaných čísel. To znamená, že nově vytvořený jedinec je s velkou pravděpodobností blízko svému rodiči. Gaussovská mutace se chová podobně, ale u PM je pravděpodobnost malých změn mnohem větší. Nový jedinec se v polynomiální mutaci vytvoří jako 
$$o = p + \delta \Delta_{max}\,$$ kde $\Delta_{max}$ je maximální velikost mutace a $\delta$ je náhodné číslo z rozdělení $$P(\delta) = \frac{1}{2}(n+1)(1-|\delta|)^n\,, \quad \delta \in (-1, 1)\,.$$ Proměnná $n$ má v tomto případě podobný význam jako u SBX a určuje tvar distribuce. Opět větší hodnota $n$ vede k větší pravděpodobnosti menších změn.

\begin{marginfigure}[-10\baselineskip]
\centering
%!TEX root = ../../main.tex
\begin{tikzpicture}
  \begin{axis}[
    axis lines=middle,
    xmin=-1,   xmax=1,
    ymin=0, ymax=5,
    xlabel=$\delta$,
    ylabel=$P(\delta)$,
    width=\marginparwidth
  ]
  \addplot [domain=-1:1,samples=100]{3/2*(1-abs(x))^2};
  \addplot [domain=-1:1,samples=100, color=red]{11/2*(1-abs(x))^10};
  
  \node[anchor=west,color=red] (n10) at (axis cs:0.08,3){$n=10$};
  \node[anchor=west,color=black] (n2) at (axis cs:0.35,0.9){$n=2$};
  \end{axis}
\end{tikzpicture}
\caption{Pravděpodobnostní rozdělení $\delta$ v polynomiální mutaci pro $n=2$ a $n=10\,.$}
\end{marginfigure}

\section{Diferenciální evoluce}

Všechny výše popsané operátory mají jednu zásadní nevýhodu a tou je, že operují s jedinci po složkách, jsou tedy vhodné především pro separabilní funkce. \emph{Diferenciální evoluce}\cite{Storn1997} přináší jiný přístup, který tímto problémem netrpí. 

V diferenciální evoluci se jedinec vytváří pomocí jediného operátoru, který v sobě kombinuje jak mutaci, tak křížení. Jeho vstupem jsou hned čtyři jedinci z populace rodičů. Jako rodič $p_4$ se většinou volí postupně všichni jedinci z populace (není tedy volen náhodně), další tři rodiče jsou zvoleny náhodně. Nový jedinec vzniká tak, že se k rodiči $p_3$ přičte rozdíl rodičů $p_2$ a $p_3$ a potom se provede křížení s rodičem $p_4$. Postup se dá zapsat jako 
$$o^i = \twopartdefotherwise{p_3^i + F(p_1^i-p_2^i)}{r^i < CR \text{ nebo } i = R}
                   {p_4^i}{\,,}$$
kde $F$ a $CR$ jsou parametry mutace a křížení, $r_i$ je náhodné číslo z rovnoměrného rozdělení $\uniform{0}{1}$, $R$ je náhodné číslo z množiny $\{1,\dots,d\}$, které zajišťuje, že v každém jedinci alespoň jedna pozice vznikne pomocí první řádky, a $o^i$ a $p_j^i$ značí $i$-tou složku potomka resp. $j$-tého rodiče.

Právě první část definice zajišťuje, že diferenciální evoluce je invariantní vůči rotacím, posunům a škálování prohledávaného prostoru a tedy funguje lépe pro neseparabilní a špatně podmíněné funkce. Pro toto chování je velmi důležitá volba parametru $CR$. Pro vysoké hodnoty $CR$ větší část jedince vzniká právě pomocí rozdílu dvou rodičů, pro nízké hodnoty $CR$ naopak je většina jedince tvořena rodičem $p_4$. Parametr $CR$ se doporučuje nastavovat na nižší hodnoty ($CR=0.2$) pro separabilní funkce a na vyšší hodnoty pro neseparabilní funkce ($CR=0.9$). 

Parametr $F$ určuje, jak velká část rozdílu dvou jedinců se použije a tedy i to, jak velké se při mutaci dělají kroky. Nejčastěji se doporučuje hodnota okolo $F=0.8$, ale vyskytují se nastavení mezi 0.5 a 2.0. Někdy se dokonce $F$ bere jako náhodná hodnota z intervalu $[0.5, 1.0]$. 

Kromě výše uvedeného operátoru se v diferenciální evoluci liší i selekce. Nový potomek se porovnává s rodičem $p_4$ a v populaci se nechá jen lepší z obou. To je i důvod, proč se často rodič $p_4$ nevolí náhodně.

V průběhu let vznikl systém pro klasifikaci různých variant diferenciální evoluce, popsaná verze se dnes nazývá \de{rand}{1}{bin}. Označení vyjadřuje, že se jedná o diferenciální evoluci, kde jsou rodič $p_3$ pro mutaci je volen náhodně, vybírá se jedna dvojice a používá se binomiální křížení. Existuje ale mnoho dalších variant\cite{Mezura-Montes:2006:CSD:1143997.1144086}. Například se místo náhodně zvoleného jedince $p_3$ bere vždy nejlepší jedinec z populace (\de{best}{.}{.}) nebo se bere více dvojic při mutaci. Pro $k$ dvojic potom vypadá operátor v diferenciální evoluci typu \de{.}{k}{.} jako
$$o^i = \twopartdefotherwise{p_3^i + F\sum_{i=1}^k(p_{1,k}^i-p_{2,k}^i)}{r^i < CR \text{ nebo } i = R}
                   {p_4^i}{\,,}$$
kde značení je stejné jako výše. 

Místo binomiálního křížení se občas používá tzv. \emph{exponenciální křížení}. To používá stejný parametr $CR$, ale jiným způsobem. Jedná se vlastně o obdobu jedno- nebo dvou-bodového křížení. Začne se na náhodné pozici v jedinci a kopírují se parametry z druhého jedince, dokud je náhodně zvolené číslo z $\uniform{0}{1}$ menší než $CR$. Pokud se narazí dříve na konec vektoru, pokračuje se znovu od začátku. Varianta s tímto křížením se nazývá \de{.}{.}{exp} a ačkoliv je velmi oblíbená, vysloužila si v poslední době kritiku\cite{Tanabe2014} právě kvůli tomu, že v křížení je větší pravděpodobnost, že se překopírují hodnoty, které jsou na sousedních pozicích a kvůli tomu je algoritmus závislý na pořadí proměnných.
