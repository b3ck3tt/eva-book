%\section{Kombinatorická optimalizace}
%
%

\section{Problém obchodního cestujícího}

Problém obchodního cestujícího (POC) je jedním z nejznámějších NP-úplných problémů. Jde o nalezení nejkratší cesty mezi $n$ městy, která navštíví všechna města právě jednou a skončí v počátečním městě~\footnote{V daném ohodnoceném úplném grafu máme najít nejkratší hamiltonovskou kružnici.}. 

Evoluční řešení problému obchodního cestujícího má zřejmou fitness --- délku cesty --- zatímco reprezentací řešení existuje několik a genetické operátory (hlavně operátor křížení) jsou na těchto reprezentacích velmi závislé. V následujícím ukážeme nejprve nejčastější reprezentaci pomocí cesty (permutace)~\footnote{Permutační zakódování jedinců je přirozené i u jiných problémů, zapamatujme si jej.}, které je sice velmi přímočará, ale vyžaduje specifické operátory křížení. O operátorech mutace a inicializace se zmíníme poté.   


\subsection{Reprezentace cestou}

Reprezentace cestou je velmi názorná, jde o posloupnost celočíselných hodnot označujících pořadí měst tak, jak je navštívíme. Jedinci evolučního algoritmu tedy permutaci o $n$ prvcích.  $(412876935)$ tedy říká, že řešení reprezentuje cestu městy v pořadí $4-1-2-8-7-6-9-3-5$. Zjevnou nevýhodou této reprezentace je, že jednobodové křížení dvou permutací typicky nevyrobí potomky jako validní permutace. 

Jednotlivé návrhy křížení proto pracují tak, aby opravily nebo doplnily jedince, který od svých rodičů zdědí vlastnosti, jež autoři považují za důležité. V různých kříženích se proto (kromě zachování permutace) zrcadlí heuristický přístup k tomu, co by měla rekombinace dvou cest v problému POC zohlednit. 

Prvním návrhem křížení je PMX --- křížení částečné shody nebo také částečného zobrazení.\footnote{Partially mapped/matched crossover} Goldberga a Linglea. Snahou tohoto křížení je zachovat co nejvíce měst na svých původních pozicích. Jde o dvoubodové křížení, které dvěma náhodnými řezy rozdělí každého rodiče na tři části, přičemž střední část použije k definici zobrazení, která města na stejných pozicích spolu korespondují. V příkladu na obr.~\ref{PMX} definujeme tedy zobrazení měst: $1\leftrightarrow 4, 8\leftrightarrow 5, 7\leftrightarrow 6, 6\leftrightarrow 7$. Křížení PMX vymění středy jedinců a dále doplní z původních cest ta města, která nezpůsobí konflikt, t.j. nejsou ve vyměněných středních částech. Nakonec se pro doplnění zbylých měst použije korespondence dané zobrazením dle střední části.   

\begin{figure}%
\begin{verbatim}
(123|4567|89) PMX (452|1876|93) :
(...|1876|..)  (...|4567|..) 
a zobrazení: 1-4 8-5 7-6 6-7 
lze doplnit (.23|1876|.9)  (..2|4567|93)
a dle zobr. (423|1876|59)  (182|4567|93)
\end{verbatim}
\caption{Příklad křížení PMX.}%
\label{PMX}%
\end{figure}

Davis navrhl jiný přístup ke křížení, tzv. křížení zachovávající pořadí (OX)~\footnote{Order crossover}, které se snaží co nejvíce zachovat relativní pořadí jednotlivých měst v cestě. Obdobně jako v případě PMX jde o dvoubodové křížení, které nejprve vymění středy jedinců. Dále přeuspořádáme cestu druhého rodiče od druhého bodu křížení a popořadě doplňujeme prvního potomka tak, že vynecháváme města, která již jsou v nové střední části. Druhý potomek vznikne stejným způsobem z přeuspořádané cesty prvního rodiče, viz obr.~\ref{OX}.   


\begin{figure}%
\begin{verbatim}
(123|4567|89) OX (452|1876|93) : 
(...|1876|..)  (...|4567|..) 
a přeuspořádáme cestu 2 od druhého bodu 
křížení: 9-3-4-5-2-1-8-7-6
vyhodíme překřížená města z 1, zbyde: 9-3-2-1-8
doplníme potomka 1: (218|4567|93)
obdobně potomek 2: (345|1876|92)
\end{verbatim}
\caption{Příklad křížení OX.}%
\label{OX}%
\end{figure}

Na základě křížení OX vznikl algoritmus křížení modifikovaného křížení (ModX)~\footnote{Modified crossover}, které se ukázalo jako úspěšné v řešení problémů rozvrhování. ModX je vlastně OX jen s jedním křížícím bodem, kde oba rodiče přeuspořádáme tak, aby cesta začínala za křížícím bodem a pak křížíme dle principu OX od začátku obou jedinců, viz obr.~\ref{ModX}.    

\begin{figure}%
\begin{verbatim}
(0845|671239) ModX (6712|483590)

potomek 1: (6712084539)
potomek 2: (0845671239) 
\end{verbatim}
\caption{Příklad křížení ModX.}%
\label{ModX}%
\end{figure}

Posledním z klasických křížení permutačních řešení POC, které si uvedeme, je cyklické křížení (CX)~\footnote{Cycle crossover} navržené Davisem a Hollandem. Snahou autorů bylo, aby každá pozice v potomkovi odpovídala, pokud možno, jednomu s rodičů. Za tímto účelem se při tvorbě potomka pohybujeme v cyklech po permutaci podle odpovídajících měst. První pozici si zvolíme náhodně z jednoho z rodičů, pak se snažíme vzít to město, které je na stejné pozici u druhého z rodičů, ale bereme z pozice v prvním rodiči, atd., dokud nedokončíme cyklus. Pak doplníme městy a pozicemi z druhého rodiče, viz obr.~\ref{CX}.  

\begin{figure}%
\begin{verbatim}
(123456789) CX (412876935)
první pozice, náhodně třeba z prvního P1=(1........), 
teď musíme vzít 4, P1=(1..4....), pak 8, 3 a 2
P1=(1234...8.), dál už nelze, doplníme z druhého
P1=(123476985)
Obdobně P2=(412856739)
\end{verbatim}
\caption{Příklad křížení CX.}%
\label{CX}%
\end{figure}


\subsection{Křížení rekombinací hran}

Hlavní myšlenkou křížení rekombinací hran (ER)~\footnote{Edge recombination crossover} je přesun pozornosti od uzlů grafu (městům) k hranám (cestám). Autoři Whitley s kolegy zjistili, že výše zmíněné operátory ve skutečnosti nezachovávají více jak 60\% hran z obou rodičů. Algoritmus ER (a jeho vylepšení ER2) se tedy snaží pracovat tak, že v potomcích zachová co nejvíce již existujících hran. Pracuje ve dvou fázích --- nejprve si vybuduje seznam sousedů všech vrcholů, a potom tvoří nové jedince s pomocí tohoto seznamu, viz alg.~\ref{alger}.

\begin{algorithm}
\caption{Křížení rekombinací hran}
\label{alger}
\begin{algorithmic}
\Procedure{ER}{} 
\State $K \gets []$ prázdný seznam
\State $N \gets $ první uzel náhodného rodiče
\While{Length(K) $<$ Length(Parent)}
\State $K \gets [K,N]$ 
\State Vyhoď $N$ ze všech seznamů sousedů
\If{seznam sousedů $N \not\=\emptyset$} 
\State 	$N^* \gets$ soused s nejméně sousedy v seznamu (nebo náhodný z více takových)
\Else 
\State 	$N^* \gets$ náhodně zvolený uzel $\notin K$
\EndIf
\State 	$N \gets N^*$
\EndWhile
\EndProcedure
\end{algorithmic}
\end{algorithm}


Obr.~\ref{ER} a~\ref{ER2} ukazují práci původní varianty algoritmu a jejího vylepšení, které dává přednost výběru měst, která jsou v seznamu dvakrát (označené \#). Dodejme, že algoritmus se může zastavit ještě před sestavením kompletní permutace, ale to se dle praktických zkušeností děje jen zřídka (v 1--1,5\% případů). Algoritmy křížení ER jsou v praxi velmi úspěšné. 

\begin{figure}%
\begin{verbatim}
(123456789) ER (412876935)
1:   9 2 4
2:   1 3 8
3:   2 4 9 5
4:   3 5 1
5:   4 6 3
6:   5 7 9
7:   6 8
8:   7 9 2
9:   8 1 6 3

Začnu v 1, následníci jsou 9, 2, 4
9 vypadává, má 4 násl., z 2  a 4 náhodně vyberu 4
násl. 4 jsou 3 a 5, beru 5, 
neboť 5 má 3 násl, zatímco 3 má 4 násl. 
Teď máme (145......), podobně pokračujeme
... (145678239)
\end{verbatim}
\caption{Příklad křížení ER.}%
\label{ER}%
\end{figure}

\begin{figure}%
\begin{verbatim}
(123456789) ER (412876935)
1:   9 #2 4
2:   #1 3 8
3:   2 4 9 5
4:   3 #5 1
5:   #4 6 3
6:   5 #7 9
7:   #6 #8
8:   #7 9 2
9:   8 1 6 3

(12.......), (128......), (1287.....), (12876....)
náhodně např. (128765...), (1287654..), (12876543.)
(128765439)
\end{verbatim}
\caption{Příklad křížení ER2.}%
\label{ER2}%
\end{figure}

\subsection{Další operátory}

Zatím jsme hovořili jen o kříženích, zastavme se nyní u dalších operátorů pro problém POC. 

Inicializaci jedinců na začátku algoritmu můžeme provést generováním náhodných permutací, ale lze využít i heuristik pro lepší výchozí řešení. Jednou z nich je jednoduchý algoritmus hladového typu, který inicializuje tak, začne náhodným městem a pak postupně vybírá nejbližšího souseda k již určenému městu. Jinou možností inicializace je algoritmus vkládání hran~viz alg.~\ref{alginr}.


\begin{algorithm}
\caption{Inicializace vkládáním hran}
\label{alginr}
\begin{algorithmic}
\Procedure{Vkládání hran}{} 
\State cesta $T \gets$ náhodná hrana
\While{Length(K) $<n$}
\State vyber nejbližší město $c$ k cestě $T$
\State najdi hranu $(k-j)\in T$ tak, že minimalizuje rozdíl mezi $(k-c-j)$ a $(k-j)$
\State Vyhoď $(k-j)$ z $T$
\State Vlož $(k-c)$ do $T$
\State Vlož $(c-j)$ do $T$
\EndWhile
\EndProcedure
\end{algorithmic}
\end{algorithm}

Mutačních operátorů vhodných pro POC je celá řada, i když nejjednodušší (změna jednoho města) naopak použít nelze. Oblíbené jsou ale mutace 
Posun podcesty, Záměna 2 měst, Záměna podcest nebo heuristiky jako je 2-opt~viz alg.~\ref{dvaopt}~\footnote{G. A. CROES (1958). A method for solving traveling salesman problems. Operations Res. 6 (1958) , pp., 791-812.}. 
Zajímavé na permutační reprezentaci také je, že u ní funguje inverze. 


\begin{algorithm}
\caption{Heuristika 2-opt pro lokální zlepšení cesty}
\label{dvaopt}
\begin{algorithmic}
\Procedure{2-opt}{cesta, i, k} 
\State new1 $\gets$ cesta[1] .. cesta[i-1]
\State new2 $\gets$ cesta[i] .. cesta[k] v převráceném pořadí
\State new3 $\gets$ cesta[k+1] .. cesta[n]
\Return new1+new2+new3
\EndProcedure
\end{algorithmic}
\end{algorithm}


\subsection{Jiné reprezentace}

Přesto, že permutační zakódování jedinců POC je nejčastější, zkoumají se i jiné možnosti reprezentace cest. 

Jednou z nich je reprezentace sousednosti~\footnote{Adjacency representation}, která cestu reprezentuje také jako seznam měst, ale 
platí že město $j$ je na pozici $i$, vede-li cesta z $i$ do $j$. Takže např. 
(248397156) odpovídá cestě 1-2-4-3-8-5-9-6-7. Každá cesta má jen 1 reprezentaci, ale některé seznamy negenerují přípustnou cestu
Klasické křížení v této reprezentaci také nefunguje (a navrhují se různé opravy), ale hlavní motivací pro její zavedení bylo, že 
v ní mají dobrý význam klasická schémata. Např  (*3*...) značí všechny cesty s hranou 2-3

Problém fungování klasického jednobodového křížení se snaží řešit ordinální reprezentace~\footnote{reprezentace bufferem}. Ta je založena na myšlence, že číslo reprezentující jedno město nebude označovat jeho absolutní název ale jen relativní pozici (pořadí) v bufferu všech měst, ze kterého při konstrukci jedince použitá města mažeme. Takže například máme-li buffer (123456789) a cestu 1-2-4-3-8-5-9-6-7, její reprezentace je (112141311).
Relativita ordinální reprezentace způsobí, že nyní můžeme dva jedince křížit jednobodově (potomci jsou syntakticky správně), ale význam takového křížení není zřejmý, jde spíše o makromutaci.

Existuje i několik pokusů o  maticové reprezentace cest jako jedinců evolučního algoritmu, většinou jde o binární matice, kde jednička na pozici $(i,j)$ znamená buď, že v cestě vede hrana $i\to j$, anebo (častěji), že město $i$ předchází v cestě město $j$. Operátory nad maticemi používají často logické operace nad prvky matic a opravné mechanismy. Většinou tyto reprezentace nedosahují výrazných výsledků. 

\section{Jiné problémy}

\subsection{Batoh}


Evoluční algoritmy jsou aplikovány i na řešení dalších kombinatorických problémů, přičemž je zajímavé, že některé problémy, ač jsou např. také NP-těžké jako POC, představují jiné překážky pro EVA. Uvažme například problém 0-1 batohu, který je definován následovně: 

Mějme množinu věcí očíslovaných od 1 do $n$, každá má hmotnost $w_i$a cenu $v_i$, a maximální nosnost batohu W. Úkolem je
maximalizovat: $$\sum _{i=1}^{n}v_{i}x_{i}$$ za podmínek $$\sum _{i=1}^{n}w_{i}x_{i}\leq W\;\;\; \mbox{a zároveň}\;\;\; x_{i}\in \{0,1\}$$.
$x_i$ representuje počet věci $i$ v batohu, my počítáme s tím, že každou věc vezmeme maximálně jednou, proto 0-1 batoh. Zakódování problému se zdá --- na rozdíl od POC --- triviální, můžeme použít bitovou mapu pro věci určené do batohu. Operátory křížení a mutace se zdají být také jednoduché. Problémem v tomto případě je fitness jedince. Za prvé chceme maximalizovat cenu, za druhé nechceme překročit kapacitu $W$, ale blížit se k ní. Lze využít multi-kriteriální EVA, ale v praxi se osvědčilo i velmi jednoduché řešení z rodiny opravných algoritmů dekódování jedince, které bývá překvapivě efektivní. Dekodér procházející bitový vektor jedince interpretuje jedničku na místě $i$ jako: dej věc $i$ do batohu, pokud by se nepřeplní. 

\subsection{Rozvrhování}

Rozvrhování je dalším těžkým problémem, kde se evoluční algoritmy osvědčují jako heuristika pro nalezení přibližných řešení. Rozvrh reprezentujeme přímočaře maticí, kde řádky odpovídají učitelům, sloupce hodinám a hodnoty jsou kódy předmětů. Mutace většinou míchají předměty, křížení mění lepší řádky z jedinců. Problémem u rozvrhování je, jak určit fitness a jak pracovat s omezujícími podmínkami. Obecně se ukazuje, že dobré je definovat fitness řádku (t.j. jak je spokojen učitel) a tu pak zkombinovat s dalšími (často měkkými) kritérii kvality rozvrhu. Existující tvrdá omezení (kdy mohou učitelé, co kde lze učit, $\dots$) je z důvodů efektivity často nutné řešit již v operátorech, aby se zbytečně negenerovala řada nepřípustných řešení.   

\subsection{Plánování výroby}
Na problému plánování výroby~\footnote{shop scheduling} si můžeme ukázat, že různá zakódování nám vyřeší různé těžkosti s řešením (a přinesou jiná). Problém je definován takto: Mějme výrobky $o_1,\dots o_N$, sestávající z částí $p_1, \dots p_K$. Pro každou část máme stejný (flow shop) nebo více (job shop) plánů, jak jí vyrobit na strojích $m_1, \dots m_M$, a stroje mají různé doby nastavení na jiný výrobek. Úkolem samozřejmě je minimalizovat čas výroby. Tento čas se také jednoduše může stát fitness. 

Evoluční řešení volí dvě cesty jak zakódovat jedince, První typ kódování je jednoduchý permutační --– vyvíjený plán je jen permutace pořadí výrobků. Dekodér pak musí zvolit plány pro jejich části. Výhody jsou jednoduchá reprezentace a možnost použít křížení např. z problému POC. Ale ukazuje se, že tento přístup není efektivní. Dekodér je nucen řešit tu komplikovanou část úlohy, výběr plánu, a také většina operátorů z POC není vhodná.
Na druhou stranu, přímá reprezentace jedince jako celého plánu (nazývá se paralelní kódování strojů) klade nároky na specializované (komplexní) evoluční operátory~\cite{Werner:2013}.
