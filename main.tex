\documentclass[12pt,fleqn,a4paper,oneside]{book}

\usepackage{makeidx}
\usepackage{graphicx}
\usepackage{algorithmicx}
\usepackage{algpseudocode}
\usepackage{algorithm}
\usepackage{color}
\usepackage{times}


\usepackage[czech]{babel}
\usepackage[utf8x]{inputenc}
%\usepackage[T1]{fontenc}     

% \usepackage[utf8x]{inputenc}  % pro unicode UTF-8
% \usepackage[cp1250]{inputenc} % pro win1250
% \usepackage[T1]{fontenc}
%\usepackage{float}
\floatname{algorithm}{Algoritmus}
\renewcommand{\listalgorithmname}{Seznam algoritmů}
\newcommand{\Real}{{\cal R}}

\newcommand{\pozn}[1]{{\textcolor{red}{\sc #1}}}
%\def\pozn#1{{\textcolor{red}{\sc #1}}}

\makeindex

\frontmatter

\title{EVA -- Přírodou inspirované prohledávací algoritmy}
\author{Roman Neruda, Martin Pilát\\
MFF UK
}
\date{\includegraphics[width=5cm]{mff_logo.png}}

\begin{document}
%%%%%%%%%%%%%%%%%%%%%%%%%%%%%%%%%%%%%%%%%%%%%%%%%%
% hack, aby nepretekaly cz slova, ktery neumi delit
\sloppy 
%%%%%%%%%%%%%%%%%%%%%%%%%%%%%%%%%%%%%%%%%%%%%%%%%%

\maketitle

\mainmatter

\chapter{Před úvodem}
\section{Jak to je?}
\pozn{osnova prednasky tak jak je ted}

Evoluční algoritmy

Jednoduchý genetický algoritmus

Teorie schémat

Reprezentace a operátory v GA

Evoluce kooperace

Evoluční strategie

Diferenciální evoluce

Particle swarm optimization

Evoluční strojové učení

Vícekriteriální optimalizace

Evoluční kombinatorická optimalizace

Ladění, řízení, metaevoluce

Teorie EVA podruhé

Evoluční programování

Genetické programování

Neuroevoluce

Memetické algoritmy

Dynamické krajiny fitness

Tabu search, scatter search

Biologicky věrnější evoluce


\section{Jak to bude?}
\section{Pok pok pokusy}
\pozn{skoro to tu muzeme smazat ne?}

\index{Genetický algoritmus}

\begin{algorithm}
\caption{Schéma evolučního algoritmu}
\label{obreva2}
\begin{algorithmic}
\Procedure{Evoluční algoritmus}{} 
\State $t \gets 0$
\State \emph{Inicializuj} populaci $P_t$ náhodně vygenerovanými jedinci
\State \emph{Ohodnoť} jedince v populaci $P_t$
\While{neplatí \emph{kritérium ukončení}}
\State 	vyber z $P_t$ rodiče $P'_{t+1}$ \emph{Rodičovskou selekcí}
\State 	\emph{Rekombinací} rodičů vzniknou potomci $P'_{t+1}$
\State 	\emph{Mutuj} potomky $P'_{t+1}$
\State 	\emph{Ohodnoť} potomky $P'_{t+1}$
\State 	\emph{Enviromentální selekcí} vyber $P_{t+1}$ z $P_t$ a $P'_{t+1}$
\State $t \gets t+1$
\EndWhile
%\Until{kritérium ukončení}
\EndProcedure
\end{algorithmic}
\end{algorithm}



\chapter{Evoluční algoritmy}
%!TEX root = main.tex
\index{Evoluční algoritmus}

%%%%%%%%%%%%%%%%%%%%%%%%%%%%%%%%%%%%%%%%%%%%%%%%%%%%%%%%%%
\section{Evoluce, geny a DNA}
%%%%%%%%%%%%%%%%%%%%%%%%%%%%%%%%%%%%%%%%%%%%%%%%%%%%%%%%%%


%%%%%%%%%%%%%%%%%%%%%%%%%%%%%%%%%%%%%%%%%%%%%%%%%%%%%%%%%%
\section{Obecné schéma evolučního algoritmu}
%%%%%%%%%%%%%%%%%%%%%%%%%%%%%%%%%%%%%%%%%%%%%%%%%%%%%%%%%%

Oblast evolučních výpočtů či algoritmů (v angličtině evolutionary computing) zastřešuje několik proudů, které se zpočátku vyvíjely samostatně. Za prehistorii této disciplíny lze považovat Turingovy návrhy na využití evolučního prohledávání z roku 1948~\cite{Turing:1948}\missingref a Bremermannovy první pokusy o implementaci optimalizace pomocí evoluce a rekombinace z roku 1962~\cite{Bremermann:1962}\missingref. Během šedesátých let se objevily tři skupiny výzkumníků, kteří nezávisle na sobě vyvíjely a navrhly své varianty použití evolučních principů v informatice. Holland publikoval v roce 1975 svůj návrh genetických algoritmů~\cite{Holland:1992}, \index{Genetický algoritmus}
zatímco skupina Fogela a spolupracovníků vyvinula metodu nazvanou evoluční programování~\cite{Fogel:1995}. 
\index{Evoluční programování}
Nezávisle na nich přišli Rechenberg a Schwefel v Německu na metodu nazvanou evoluční strategie~\cite{Beyer:2002}.
\index{Evoluční strategie}
Až do přelomu osmdesátých a devadesátých let existovaly tyto směry bez výraznější interakce, ale poté se spojily do obecnější oblasti evolučních algoritmů. V té době  Koza vytváří metodu genetického programování, Dorigo publikuje disertaci s návrhem mravenčích optimalizačních algoritmů a vznikají první pokusy o aplikaci evoluce na vývoj umělých neuronových sítí. 

U zrodu různých variant evolučních algoritmů stála inspirace přírodními fenomény, konkrétně jde o Darwinovu teorii přírodního výběru a zjednodušené principy genetiky, které poprvé načrtl Mendel. Z genetiky se evoluční algoritmy inspirují diskrétní reprezentací genotypu, z biologické evoluční teorie používají Darwinovu myšlenku o výběru jedinců v prostředí s omezenými zdroji, který závisí na míře přizpůsobení se jedinců danému prostředí.

Základní obecnou myšlenku evolučních algoritmů lze vyjádřit následujícím způsobem. Mějme populaci jedinců v prostředí, které určuje jejich úspěšnost --- fitness. Tito jedinci navzájem soupeří o možnost reprodukce a přežití, která závisí právě na hodnotě fitness. Jde tedy o množinu kandidátů na řešeni problému definovaného prostředím. Způsoby reprezentace jedinců, jejich výběru a rekombinace závisí na konkrétním dialektu evolučních algoritmů, které probereme vzápětí. 

\begin{algorithm}
\caption{Schéma evolučního algoritmu}
\label{obreva}
\begin{algorithmic}
\Procedure{Evoluční algoritmus}{} 
\State $t \gets 0$
\State \emph{Inicializuj} populaci $P_t$ náhodně vygenerovanými jedinci
\State \emph{Ohodnoť} jedince v populaci $P_t$
\While{neplatí \emph{kritérium ukončení}}
\State 	vyber z $P_t$ rodiče \emph{Rodičovskou selekcí}
\State 	\emph{Rekombinací} rodičů vzniknou potomci
\State 	\emph{Mutuj} potomky
\State 	\emph{Ohodnoť} potomky
\State 	\emph{Enviromentální selekcí} vyber $P_{t+1}$ z $P_t$ a potomků
\State $t \gets t+1$
\EndWhile
%\Until{kritérium ukončení}
\EndProcedure
\end{algorithmic}
\end{algorithm}


Základní princip fungování evolučních algoritmů je tedy následující~\cite{Eiben:2003}. Na začátku algoritmu vygenerujeme (nejčastěji náhodně) první iniciální populaci jedinců. Všechny jedince v populaci ohonotíme ohodnocovací funkcí. Hodnota této funkce určuje šanci výběru jedinců během rodičovské selekce. Vybraní jedinci jsou potom rekombinováni pomocí rekombinačního operátoru, který typicky ze dvou jedinců vytváří jednoho či dva potomky, a pomoci operátoru mutace, který typicky provádí drobné změny jednoho jedince. Tímto postupem si vytvoříme množinu nových kandidátů řešení, a tito noví jedinci potom soutěží s původními jedinci o místo v nové populaci. Výběr jedinců do nové populace (tedy jakési slití rodičů a potomků) má na starosti enviomentální selekce beroucí v úvahu fitness jedinců a připadně další ukazatele jako je například stáří jedinců. Tím je vytvořena nová generace a tento cyklus pokračuje do splnění určitého kritéria ukončení, což je nejčastěji dostatečně dobrý nejlepší jedinec nebo předem určený počet generací.



\chapter{Genetické algoritmy}
%%%%%%%%%%%%%%%%%%%%%%%%%%%%%%%%%%%%%%%%%%%%%%%%%%%%%%%%%%
\section{Genetické algoritmy}
%%%%%%%%%%%%%%%%%%%%%%%%%%%%%%%%%%%%%%%%%%%%%%%%%%%%%%%%%%
\index{Genetický algoritmus}

Genetické algoritmy jsou asi nejznámější součástí evolučních výpočtů a v~různých obměnách se používají hlavně při řešení optimalizačních úloh. Je zajímavé, že původní Hollandovou motivací při návrhu genetického algoritmu bylo studovat vlastnosti přírodou inspirované adaptace~\cite{Holland:1992}. Velká část původní literatury byla věnována popisu principů, jak genetický algoritmus pracuje při hledání řešení úlohy. Zajímavé jsou paralely s matematickým problémem dvourukého bandity, který je příkladem na udržování optimální rovnováhy mezi explorací a exploatací. 

Nejjednodušší varianta genetického algoritmu pracuje s binárními jedinci, to znamená, že parametry řešené úlohy je nutno vždy zakódovat jako binární řetězce. Tento přístup je výhodný z hlediska jednoduchosti použitých operátorů, ale binární zakódování na druhou stranu nemusí být nejvhodnější reprezentací problému. Způsob fungování jednoduchého genetického algoritmu je také poměrně jednoduchý. Algoritmus přechází mezi populacemi řešení tak, že nová populace zcela nahradí předchozí. Výběr rodičů je často realizován tzv. ruletovou selekcí, která vybírá jedince náhodně~s pravděpodobností výběru úměrné jejich fitness. Rekombinačním operátorem je jednobodové křížení, které náhodně zvolí stejnou pozici v rodičích a vyměňí jejich části. Pravděpodobnost uskutečnění operace křížení je jedním z parametrů programu a obvykle je poměrně vysoká (0,5 i více). Mutace provádí drobné lokální změny tak, že prochází jednotlivé bity řetězce a každý bit s~velmi malou pravděpodobností změní. Pravděpodobnost mutace je typicky nastavena, tak aby došlo průměrně ke změně jednoho bitu v populaci (oblíbená dolní mez) nebo v jedinci (horní mez). 

\begin{algorithm}
\caption{Schéma Hollandova gentického algoritmu}
\label{obrga}
\begin{algorithmic}
\Procedure{Jednoduchý genetický algoritmus}{} 
\State $t \gets 0$
\State \emph{Inicializuj} populaci $P_t$ $N$ náhodně vygenerovanými binárními jedinci délky $n$
\State \emph{Ohodnoť} jedince v populaci $P_t$
\While{neplatí \emph{kritérium ukončení}}
\For{$i \gets 1, \dots, N/2$}
\State 	vyber z $P_t$ 2 rodiče \emph{Ruletovou selekcí}
\State 	S pravděpodobností $p_C$ \emph{Zkřiž} rodiče
\State 	S pravděpodobností $p_M$ \emph{Mutuj} potomky
\State 	\emph{Ohodnoť} potomky
\State  Přidej potomky do $P_{t+1}$
\EndFor
\State 	Zahoď $P_t$
\State $t \gets t+1$
\EndWhile
\EndProcedure
\end{algorithmic}
\end{algorithm}

Ruletovou selekci si dle metafory můžeme představit tak, že kolo rulety rozdělíme na výseče odpovídající velikostí hodnotám fitness jedinců a při výběru pak $n$ krát vhodíme kuličku. Často používaným vylepšením ruletové selekce je tzv. stochastický univerzální výběr, který hodí kuličku do rulety jen jednou a další jedince vybírá deterministicky posunem pozice kuličky o $1/n$. Tento výběr pro malá $n$ lépe aproximuje ideální počty zastoupení jedinců v další generaci. Dalšími varianty rodičovské selekce nepracují s absolutními hodnotami fitness, ale vybírají náhodně v závislosti na pořadí jedince v populaci setříděné podle fitness, což zanedbává absolutní rozdíly mezi hodnotami. Další variantou rodičovské selekce je tzv $k$-turnaj, kdy nejprve vybereme $k$ jedinců náhodně a z nich pak vybereme nejlepšího. 

V současnosti se oblast genetických algoritmů neomezuje jen na binární kódování jedinců, časté je celočíselné, permutační nebo reálné kódování, která ale vyžadují specifické operace křížení a mutace~\cite{Michalewicz:1996, Mitchell:1996}. O některých se zmíníme dále. 


\chapter{Evoluční strategie}



\backmatter

\tableofcontents
\printindex
\listoffigures
\listoftables
\listofalgorithms 

\nocite{*}
\bibliographystyle{plain}
\bibliography{evabook}

\end{document}
